Here, a minimal amount of background information is presented to clarify some of our design decisions. 

\subsection{Energy-harvesting Devices}

%1. why \textit{small form factors EH sensor nodes}\\
\noindent\textbf{ Why miniturized sensors.} A small sensor is a less intrusive device and compared to a bigger one; therefore, many applications prefer sensors of small form factors.
% (imagine the difference in the implications of embedding a sensor of the size of 2 AAA batteries and a sensor of a few cubic millimeters in volume in a shoe for step counting, for example).  
Long sensors' lifetimes lead to less maintenance costs, and therefore, many applications demand sensors with long liftimes. However, long lifetime and small form factor are conflict goals.
% 2. classify EH with continuous power and tiny EH with intermittent power \\
For example, rechargeable batteries paired with energy harvesters can continuously power sensor nodes for a relatively long time. But, 
they obviously increase nodes' sizes and costs. Additionally, rechargeable batteries wear out after a few hundreds of charging cycles, which means
they also limit sensors' lifetimes~\cite{aditya2008comparison}.  
%
However, if an application's requirements put hard constraints on the size of the sensors, then removing the batteries is one of the first options to be considered. 
Battery-less energy-harvesting sensors have the potential to operate for very long time (decades long) while being tiny; but, they operate intermittently.  
%
They charge a small capacitor to guarantee uninterrupted operations for a certain minimum duration. 
This guarantee facilitates the software architecture design. An intermittent program is usually decomposed  into a chain of small atomic regions that  are  glued together with checkpoints~\cite{ransford2011mementos}. 
%Once, the capacitor has been depleted, the sensor powers down, letting the energy-harvester accumulates energy again. 

%
\noindent\textbf{Do big capacitors enable continuous operation?}. Big capacitors allow longer operational periods, but they also need more time to charge. 
As such, the on/off duty cycle of a big or small capacitor may still be the same. Additionally, very big capacitors might violate
the size requirements of an application. For example, the WISPCam---an RF powered camera---~\cite{naderiparizi2015wispcam} uses a supercapacitor that is about 50 times the size of the original WISP's capacitor---a computational RFID tag---~\cite{smith2006wirelessly}.

\todo{charging big capacitors vs small ones}
\todo{cold start of big capacitors vs small ones}
%Moreover, charging a bigger capacitor requires higher input voltage, because the input voltage must be higher than the voltage that has been accumulated in the capacitor to continue the charging. 
%This phenomena makes charging big capacitors using tiny energy harvester less efficient~\cite{buettner2011dewdrop}. 

%boosting ambient energy using an energy-conditioning circuit is possible on the expense of device complexity, form factor, energy consumption, and cost. 

\subsection {Speech types}
%
\noindent\textbf{Why isolated word speech type}. Speech recognition algorithms can be classified based on the type of speech that they can recognize into \textit{spontaneous speech, continuous speech, connected word,} and \textit{isolated word}~\cite{gaikwad2010review}.
Systems with \textit{continuous} or \textit{spontaneous speech} recognition are the closest to natural speech but are the most difficult to create because they need special methods to detect words boundaries~\cite{gaikwad2010review}. This is less the case for the \textit{connected word} type, where a minimum pause between the words is required. The type with the least complexity is the \textit{isolated word} type. It requires a period of silence on both sides of a spoken word and accepts only single words. 
 
Voice is a natural way for the human to interact with small devices. However,
implementing speech recognition on resources---memory, computation power, and energy---limited platforms is challenging, to say the least. Therefore, we attempt to recognize, with our command recognizer prototype, the simplest type of speech, isolated words. 