The main obstacles to achieve truly ubiquitous sensing are (i) the limitations of battery technology - batteries are short-lived, hazardous, bulky, and costly - and (ii) the unpredictability of ambient power. The latter causes sensors to operate intermittently, violating the availability requirements of many real-world applications. 

In this paper, we present the \textit{\fullcis} (\cis), an intermittently powered ``sensor'' that senses continuously!
Our key observation being that, the power cycles of energy-harvesting battery-less devices do not show correlation even when they are drawing energy from the same source, running the same application, and in close proximity.
We challenged our observation using different real hardware and energy sources and showed that this observation holds.

Another important finding is that a \cis designed for certain (minimal) energy conditions requires no explicit spreading of awake times due to embedded randomness in the powering subsystem. However, when the available energy exceeds the design point, nodes employing a sleep mode (to extend their availability) do wake up collectively at the next event. This synchronization leads to problems as multiple responses will be generated, and -worse- subsequent events will be missed as nodes will now recharge at the same time.
To counter this unwanted behavior we designed an algorithm to estimate the number of active neighbors and respond proportionally to an event. 
We show that when intermittent nodes randomize their responses to events, in favorable energy conditions, the \cis reduces the duplicated captured events by 50\% and increases the percentage of capturing entire bursts above 85\%. 
