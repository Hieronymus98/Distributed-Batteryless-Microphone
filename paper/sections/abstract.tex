%
Small sensors are desired. As battery size shrinks, its stored energy reduces too. This imposes regular maintenance of devices otherwise functional. Sensors have to forgo batteries and rely on perpetual energy sources, such as light, to enable long-term affordable sensing. Battery-less sensors, however, operate intermittently, when ambient energy is available. Intermittent operation prevents wide adaptation of battery-less sensors as it does not meet the requirements of many real world applications. 

To bridge this gap we propose the concept of a distributed intermittent system---the abstraction of a group of intermittently powered devices (or nodes). We hypothesize that as the number of intermittent devices increase their collective up time approaches continuous time. However, if power cycles of intermittently powered devices are correlated, then they may tend to cluster and the benefit of adding nodes demolish, unless special techniques are applied. 

%The on/off cycles of intermittent devices can be correlated or not, depending mainly on their relative locations and running applications. If sensors' power cycles are uncorrelated, then adding more sensors to a distributed intermittent system enables longer observation timespan. Correlated power cycles, on the other hand, tend to cluster, around the mean of their distribution, and therefore adding more sensors is of less benefit unless special technique is applied. 

This paper investigates distributed intermittent systems, proposes techniques to control the overall on-time of distributed systems, and demonstrates the first distributed intermittent system, namely, a distributed intermittent microphone. 

