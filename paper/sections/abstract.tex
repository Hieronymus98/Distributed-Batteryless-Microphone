\textcolor{red}{The vision of ubiquitous sensing runs into the reality of battery technology. Batteries are short-lived, hazardous, bulky, and costly. Battery-less sensors choose an alternative, ambient energy. Ambient power, however, is marginal and unpredictable. Consequently, battery-less sensors operate intermittently. Sporadic sensing does not meet a wide range of real-world applications; therefore, intermittent sensors have not been widely adopted.} 
%
The main obstacles to achieve truly ubiquitous sensing are the limitations of battery technology - batteries are short-lived, hazardous, bulky, and costly - and the unpredictability of ambient power. The latter causes sensors to operate intermittently and failing to meet the requirements of real-world applications. In this paper, we
%
%We 
present a \textit{\fullsys} (\sys) an intermittently powered sensor that senses continuously! 
%
\textcolor{red}{To design the \sys, we took a deep dive into the behavior of intermittent nodes.}
%
To design the \sys, we modeled the behavior of intermittent nodes.
%
We modeled their collective on-time (\sys availability) and validated this model on real hardware. We developed an algorithm for intermittent timing that enables intermittent nodes to estimate their  charging time, without requiring additional hardware. Using its on-time as an ambient energy meter, an intermittent node can estimate the number of its active counterparts. This enables \sys to be ambient energy aware and overcome the \textit{hibernating-power-state} and the \textit{overpowering} problems. Finally, we realize the \sys in a prototype of a batteryless voice assistant agent. Despite $x\%$ duty cycle of an intermittent node, the \fullcim recognized $y\%$ commands
%
