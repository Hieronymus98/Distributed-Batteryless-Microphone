The main obstacles to achieve truly ubiquitous sensing are (i) the limitations of battery technology - batteries are short-lived, hazardous, bulky, and costly - and (ii) the unpredictability of ambient power. The latter causes sensors to operate intermittently, violating the availability requirements of many real-world applications. 

In this paper, we present the \textit{\fullsys} (\sys), an intermittently powered ``sensor'' that senses continuously! The key idea being the use of multiple intermittent nodes to ensure that (at least) one is on at all times, providing a sense of continuous availability.
To establish the required number of nodes, we modeled -and validated on real hardware- the specific (dis)charge behavior of individual nodes over time, and the emerging collective availability. 

An important finding is that a \sys designed for certain (minimal) energy conditions requires no explicit spreading of awake times due to randomness in the power source and node hardware. However, when the available energy exceeds the design point, nodes employing a sleep mode (to extend their availability) do wake up collectively at the next event. This synchronization leads to problems as multiple responses will be generated, and -worse- subsequent events will be missed as nodes will now recharge at the same time.
To counter this unwanted behavior we designed an algorithm to estimate the number of sleeping neighbors and respond proportionally to an event. 
We show that when intermittent nodes randomize their responses to events, in favorable energy conditions, the \sys reduces the duplicated captured events by 50\% and increases the percentage of capturing entire bursts above 90\%. 
