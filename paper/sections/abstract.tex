The vision of ubiquitous sensing runs into the reality of battery technology. Batteries are short-lived, hazard, bulky, and costly---costs include manufacturing, replacement, and disposing. Batteryless sensors power themselves from ambient energy. Ambient energy is marginal and unpredictable. Consequently, batteryless sensors operate intermittently. Sporadic sensing does not meet a wide range of real-world applications; therefore, intermittent sensors have not been widely adopted. We present a \textit{\fullsys} (\sys) an intermittently powered sensor that senses continuously! To design the \sys, we took a deep dive into the behavior of intermittent nodes. We modeled their collective on-time (\sys availability) and validated this model on real hardware. We developed and an algorithm for intermittent timing that enables intermittent nodes to time their off-time without additional hardware. By measuring the on and off times, intermittent nodes make an educated guess about the environment and through that about their neighbors. This enables \sys to be environmentally aware and overcome the overpowering problem. Finally, we realize the \sys in a prototype of a batteryless voice assistant agent. Our results show that x\% of the commands are captured \todo{uptime results}.
%
