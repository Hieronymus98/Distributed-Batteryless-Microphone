% related problems: % - speech recognition on embedded devices
% - intermittency
% - parallel speech recognition

%Current speech recognition solutions tend to improve the quality by performing a large part of the computation on remote servers \cite{delaney2002low}. Using intermittently powered, battery-less devices could offer a green and maintenance-free solution, especially in places where a limited vocabulary size is needed e.g. data entry applications in smart homes \cite{mcloughlin2008speech}.

%Several attempts have been done by others to implement speech recognition on embedded devices, but most either require a separate digital signal processing unit \cite{Cheng2011} \cite{Mathew2003} or significantly higher resources \cite{delaney2005energy} \cite{Huggins-Daines2006PocketSphinx:Devices}. Both points make it very hard, if not impossible, to use these existing implementations on intermittently powered devices.

%Research on intermittently-powered devices is relatively novel and till now has focused primarily on simple sensor nodes \todo{add source}.

%The challenge for battery-less speech recognition is the fact that speech is continuous in nature. The use of multiple intermittently powered devices to perform speech recognition could offer a solution to the intermittency problem.

%Here we provide the minimum amount of information needed to make the rest of the paper self-contained. The related work section~\ref{sec:relatedWork} put the paper into context. 
\subsection{Energy harvesting}
Ambient energy is volatile, and scarce. For example, radio waves harvestable power varies from \si{\nano\watt}-scale when harvesting ambient RF energy to \si{\uW}-scale when harvesting a dedicated RF signal; and solar power ranges from tens of \si{\uW} to tens of \si{\mW} when it is harvested by a solar panels of a few \si{\cm^2} illumination surface~\cite{lucia2017intermittent,rao2017ambient}.%Mechanical harvesters deliver power on a scale ranges from \si{\nW} to \si{\W} depending on the used harvester, i.e. buttons, sliders, or knobs~\cite{} 

A tiny energy-harvesting device slowly charges its energy buffer (e.g. capacitor) while the device is off. Once the buffer is full, the device begins operating and depleting---since power consumption is much higher than harvested power---its energy reservoir until it powers down. This charging-discharging cycle repeats indefinitely (Figure~\ref{fig:powerCycle}). 

\subsection{Intermittent Computing}
% What is the problem that requires intermittent execution
Intermittent execution models~\cite{van2016intermittent,colin2016chain,lucia2015simpler,bhatti2017harvos} enable applications to progress despite frequent power failure. They decompose an application into several small code pieces and save the progress state of the application on the transitions between these code segments. Therefore, intermittent applications do not return to the same execution point (e.g. \texttt{main()}) after each power failure, as the applications that assume continuous power supply, instead they resume from the last saved progress state of execution.   
% means of intermittent execution 
\subsection{Spoken words recognition}
Speech recognition algorithms can be classified based on the type of speech that they can recognize into: \textit{spontaneous speech, continuous speech, connected word,} and \textit{isolated word}~\cite{gaikwad2010review}.

Systems with \textit{continuous} or \textit{spontaneous speech} recognition are the closest to natural speech, but are the most difficult to create because they need special methods to detect word boundaries~\cite{gaikwad2010review}. This is less the case for the \textit{connected word} type, where a minimum pause between the words is required.
 The type with the least complexity is the \textit{isolated word} type. It requires a period of silence on both sides of a spoken word and accepts only single words. 

Speech recognition consists of several steps. The basic steps are mentioned briefly here:
\textit{Speech recording and signal digitization}---a microphone records the sound waves and an ADC converts the microphone signal into a digital signal. A sampling rate of about 8 kHz is required to capture the frequencies of a human voice (100-4000Hz \cite{Bernal-Ruiz2005}). \textit{Framing}---after that the digitized signal is divided into blocks of usually 10-30 ms~\cite{gaikwad2010review,delaney2002low,delaney2005energy} called frames. \textit{Features extraction}---for each frame a feature vector is extracted containing all the relevant acoustic information. \textit{Feature matching}---finally the extracted features are matched against features known to the recognizer. 


